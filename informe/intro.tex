\section{Introducción}

El presente trabajo práctico consiste en aplicar técnicas de Métodos Numéricos y Data Science, en particular Regresiones Lineales con Cuadrados Mínimos sobre un gran conjunto de datos buscando proveer información descriptiva y de modelos que puedan ser utilizados para predecir fenómenos que afecten a la puntualidad (OTP), pero no necesariamente limitados a ésta.

Los datos a analizar comprenden cierta información relacionada a vuelos realizados en Estados Unidos entre los años 1987 y 2008, incluyendo información de la compañía, fecha y horarios planificados de partida/arribo, horarios reales de salida/llegada, causa del delay, si fueron cancelados o no, y su respectiva causa, el tipo de avión utilizado, tiempo de vuelo, tiempos de taxi, entre otras cosas. Para mas detalle puede consultarse http://stat-computing.org/dataexpo/2009/the-data.html .%PONER COMO LINK

Se tuvo en cuenta distintas funciones para predecir y se analizó cuál predice mejor utilizando una parte de los datos como entrenamiento y la otra para comparar con lo predicho. El criterio de comparacion entre funciones es analizar el ECM (Error cuadrático medio) de lo predicho con lo real.

% Complicacion con la semana 53, usamos 

