\section{Conclusión}

En cuanto a cantidad de vuelos por año pudo observarse un aumento a lo largo del tiempo. La precisión de las predicciones se vio afectada por unos outliers periódicos que ocurren cada febrero, estos elevan el error cuadrático medio creando una disrupción en el entrenamiento y por ende, en la predicción. También se observó que un entrenamiento de mayor rango trae mejores resultados en la predicción pero como los vuelos pueden estar afectados por factores externos e impredecibles, es posible que un rango demasiado grande resulte perjudicial, como podría ser el caso de tomar como parte del entrenamiento los vuelos durante los años 2001 y 2002 donde la cantidad de vuelos se ve comprometida por los eventos sucedidos.

Es importante tener en cuenta la naturaleza de la información a manejar. Se llegó a la conclusion de que es necesario tener cautela qué porción de los datos elegir para ser tenida en cuenta. El conocimiento y análisis en la información disponible permite poder ajustar lo mejor posible el entrenamiento, conociendo los datos es posible llegar a una mejor discriminación de cuales son outliers y cuáles no para que pueda resultar en una predicción más acertada.

En este caso conocer que se trata de vuelos dentro de Estados Unidos permite realizar algunas suposiciones que merecen ser tenidas en cuenta a la hora de predecir, como por ejemplo, que la cantidad de vuelos pueda ir en aumento pero se vea obstaculizada por factores que todavía no hacen efecto en el conjunto de datos pero que pueden estar presentes como la saturación del espacio aéreo o la capacidad de los aeropuertos.

Por otro lado, en cuanto a la demora de vuelos se observó que entre los años 2003 y 2005 se producían cambios similares, pero que a partir de ese último año las demoras ocurridas mostraron un patrón diferente, lo cual causaría que el entrenamiento realizado con los primeros años falle en la predicción. Sin embargo se obtuvo una predicción cercana a los hechos reales a causa de la función utilizada para predecir y de haber tenido en cuenta el origen de los datos y su comportamiento en general.

El hecho de tener en cuenta el origen y el comportamiento de los datos está basado en la hipótesis planteada, en la cual se propuso que las causas de demora afectan todas por igual y luego de haber realizado las experimentaciones pertinentes, se concluyó que la hipótesis era correcta para la predicción de demoras en los vuelos.