\section{Vuelos por mes}

Para el primer análisis se decidió observar la cantidad de vuelos por mes entre los años 2003 y 2008. En el caso de lograr una predicción ajustada, esta información podría ser utilizada para calcular insumos necesarios en los próximos años dependiendo de la cantidad de vuelos que se realizarán.

\subsection{Primeras hipótesis y bases del análisis}

Nuestra hipótesis respecto de este eje es que la cantidad de vuelos de cada aeropuerto debería aumentar a lo largo de los años dado que el tráfico áereo aumenta a la par. Los experimentos comprenden a todos los aeropuertos.

Consideramos los datos a partir del año 2003 debido al incidente de 2001 previamente mencionado.

Una observación al respecto de este gráfico es que en todos los años, febrero tiene un pico bajo de vuelos. No pudimos encontrar la razón por la cual sucede esto, lo único que pudimos observar es que esa fecha coincide con el intervalo de clases entre dos períodos de vacaciones: el president’s day y spring break.

\subsection{Datos concretos y estimaciones}

Para realizar los gráficos acumulamos por cada mes de cada año la cantidad de vuelos sucedidos. La función utilizada con CML para esta estimación fue:

\smallskip

$y(x) = ax + b|cos(x)| + c|sin(x)| + d \times log(x+1) + e + f \times cos(x) + g \times sin(x) + hx^4 + ix^3 + jx^2$

Uno de los experimentos realizados fue variar los años de entrenamiento y estimar los siguientes años. Se comparó un conjunto entrenado con un año extra que el resto para comprobar si la diferencia en la precisión era sustancial o no. La hipótesis es que al haber entrenado con una mayor cantidad de datos, es posible dar una mejor predicción de los siguientes años, pero podría ser posible que esto no ocurriese así sino que durante ese año extra la disposición de los datos sea muy particular y que se produzca el fenómeno de overfitting, resultando en una predicción de menor calidad.

Estos son los resultados de entrenar a la función entre los años 2003 y 2006, prediciendo de 2006 a 2008.

\includegraphics[scale=0.9,natwidth=732,natheight=415]{vuelos_3-6_7-8.png}

El error cuadrático medio para estas estimaciones fue de 840810926.93.

Luego se experimentó tomando distintos rangos de entrenamiento: 2003-2005, 2004-2006 y 2005-2007, prediciendo 2005-2006, 2006-2007 y 2007-2008, respectivamente.

\includegraphics[scale=0.9,natwidth=732,natheight=415]{vuelos_multiples.png}

Los EMC fueron 11228017738.2, 1215421229.42, 1214577235.23, según el orden en que los rangos están dispuestos.

Como puede observarse, el EMC para la función entrenada desde 2003 a 2006 es mucho menor que el resto, y esto se cree que se debe a ese año extra que tiene de entrenamiento.
