\section{Demoras por mes y distancia}

Decidimos analizar cómo se relacionan las demoras y la distancia del vuelo. Estas demoras tienen distintas causas, prestamos atención a las que están representadas en los dato, quedándonos con las siguientes tipos de demora:
\begin{itemize}
\item Aerolínea (problemas de mantención del avión, o de los empleados)
\item Mal clima
\item NAS(National Aviation System, demoras causadas por operaciones del aeropuerto, tráfico voluminoso y/o control del tráfico aéreo)
\item Avión retrasado (demora por un vuelo anterior llegando tarde, causando que el vuelo presente salga más tarde)
\item Seguridad
\end{itemize}

\subsection{Primeras hipótesis y bases del análisis}

Nuestra hipótesis respecto a este eje es que la causas de demoras afectan a los tipos de vuelo por igual, ya que las causas anteriormente nombradas no parecerían tener algún tipo de relación con la distancia del vuelo.

Para estos experimentos utilizamos los datos entre los años 2003-2008 (no hay información para años previos). 

%-Grafico 1
\includegraphics[scale=0.9,natwidth=732,natheight=415]{distancia1.png}

Podemos ver como la cantidad de vuelos según tipo es bastante similar, y escala muy parecido en el tiempo, salvo los vuelos cortos que sufre una caída más notoria a partir del 2008.

%-Grafico 2
\includegraphics[scale=0.9,natwidth=732,natheight=415]{distancia2.png}

En este segundo gráfico notamos, primero que el porcentaje es muy similar para las 3 categorías, hay menos vuelos retrasos en vuelos cortos, posiblemente debido a que necesitan una menor preparación, aunque no es una diferencia significativa. Y otra dato que extrajimos de este gráfico es que las demoras son estacionarias, y tienen una frecuencia bastante regular ya que podemos notar que los altibajos se producen en las mismas épocas del años para los distintos tipos. Por esto prestaremos atención al estimar a los senos y cosenos.


\subsection{Datos concretos y estimaciones}

Primero queríamos ver cómo responden las estimaciones, así que siguiendo el approach anterior, estimamos el porcentaje de demoras para vuelos medianos, tratando de estimar 2006, 2007 y 2008, para cada año tomamos los dos años anteriores

%Grafico 3 y 4
\includegraphics[scale=0.9,natwidth=732,natheight=415]{distancia3.png}
\includegraphics[scale=0.9,natwidth=732,natheight=415]{distancia4.png}

Los errores cuadráticos medios para estas estimaciones fueron 0.001143 (izquierda), y 0.001758 (derecha). La función utilizada con CML para esta estimación fue:

\bigskip

$f(x) = a + bx +c cos(x*d+e) + f * sin(x*g+h) + i$

La intuición que tuvimos para encontrar la función fue usar cosenos y senos por la frecuencia y amplitud de los datos.

Este otro experimento consistió en reutilizar los parámetros obtenidos por el método en el tipo de vuelo anterior y tratar de estimar el porcentaje de demoras para vuelos largos. 

%Grafico 5
\includegraphics[scale=0.9,natwidth=732,natheight=415]{distancia5.png}

Como se puede observar la predicción se comporta bastante parecida a los datos reales, esta se acopla muy bien a la frecuencia, pero en el 2007 se puede observar una amplitud de los datos más pequeña que los años anteriores y ahí es donde más falló la estimación.